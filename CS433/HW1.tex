\documentclass[letter,11pt,leqno]{article}
%define the title
\usepackage{amsmath}
\usepackage{fancyhdr}
\pagestyle{fancy}
\rhead{pengliu2}
\lhead{Peng Liu}
\author{Peng Liu(pengliu2)}
\title{Solutions for Homework 1}
\begin{document}
% generates the title
\maketitle
% insert the table of contents
%\tableofcontents
\section*{Problem 0 (3 points)}
\subsection*{0.0)} You are offered old homework solutions by a student who formerly took this class. You should:\\
a. Accept the solutions because no one will ever know.\\
b. Not even look at the solutions because that is the honor code set for this class.\\
c. Accept the solutions because you don’t really want to learn anything in this class. You are simply taking it to fulfill a requirement.

\subsection*{Answer:} b

\subsection*{0.1)} While working with a partner on a homework assignment, one of your classmates asks if he/she can also work with you. You should:\\
a. Tell the other student that you already have a partner for this homework assignment.\\
b. Agree to work with the other student, but only on the portion of the homework you have not yet completed.\\
c. Apologize for having found a partner too early and offer to send the other student a completed copy of your homework.

\subsection*{Answer:} a,b

\subsection*{0.2)} Suppose you and your partner are overwhelmed with work at the time the homework is due. You should:\\
a. Have your partner do half the problems, you do the other half, and combine your solutions.\\
b. Complain about your course load to everyone you meet.\\
c. Think ahead and email Professor Adve for an extension within 48 hours of when the homework is handed out.

\subsection*{Answer:} c

\section*{Problem 1 (4 points)}

\subsection*{Part (A) [2 points]}

Assume a new execution mode called “enhanced mode” provides a 1.5x speedup to the sections of programs where it applies. What percentage of a program (measured by original execution time) must run in enhanced mode for an overall speedup of 10%?

\subsection*{Answer:}
From the question, we know $Speedup_{enhanced} = 1.5$, $Speedup_{overall} = 1.1$

Then from Amdahl's Law, 
\begin{eqnarray*}
  Speedup_{overall} = \frac{1}
  {(1-Fraction_{enhanced}) + \frac{Fraction_{enhanced}}{Speedup_{enhanced}}}
\end{eqnarray*}

We get
\begin{eqnarray*}
  1.1 = \frac{1}{1-Fraction_{enhanced}+\frac{Fraction_{enhanced}}{1.5}}
\end{eqnarray*}
\begin{eqnarray*}
  Fraction_{enhanced} = \frac{3}{11}
\end{eqnarray*}

\subsection{Part(B)[2 points]}Two enhancements are proposed: one that can enhance 40% of execution time with a speedup of 1.5, and another that can enhance a different 25% of execution time with some greater speedup value. Only one of these two can be implemented. How much of a speedup is necessary in the second enhancement to give a better enhancement than the first?

\subsection{Answer:}

The overall speedup for the first case is

\begin{eqnarray*}
  Speedup_{enhanced1} = \frac{1}
  {(1-0.4)+\frac{0.4}{1.5}}\\
  = \frac{15}{13}
\end{eqnarray*}

So the speedup in the second case should satisfy

\begin{eqnarray*}
  \frac{15}{13} = \frac{1}
       {(1-0.25)+\frac{0.25}{Speedup_{enhanced2}}}
\end{eqnarray*}

So the answer is $Speedup_{enhanced2}=\frac{15}{7}$

\section*{Problem 2 (3 points)}
Consider a 2.16 cm2 die for a 64-bit processor manufactured from a 30cm-diameter wafer costing \$5000. Assume a wafer yield of 98\%. Use the defect model from the text with 0.4 defects per cm2, and α=4. What is the expected cost per die (before testing)? Ignore edge effect correction.

\subsection*{Answer:}

\begin{eqnarray*}
Cost of die = \frac{Cost of wafer}{Dies per wafer\times{}Die yield}
\end{eqnarray*}

where $Cost of wafer=5000$, $Dies per wafer=\frac{\pi\times(30/2)^2}{2.16}-\frac{\pi\times30}{\sqrt{2\times2.16}} = 281.90$

\begin{eqnarray*}
  Die yield = 0.98 \times (1+\frac{0.4\times2.16}{4})^{-4} = 0.448
\end{eqnarray*}

So $Cost of die = \frac{5000}{281.90\times0.448} = 39.59$

\section*{Problem 3 (2 points)}Suppose a processor uses 95W of power while operating at 3GHz, of which 3/4 is dynamic power. Suppose we want to run the same processor at a higher frequency which requires increasing the operating voltage proportionally as well. If the processor speed is increased by 20\%, what is the new dynamic power consumption?

\subsection*{Answer:}

\begin{eqnarray*}
  \frac{Power_{new}}{Power_{old}} = \frac{Voltage\times(1+0.2)^2\times(Frequency\times(1+0.2)}
       {Voltage^2\times{}Frequency}\\
       =1.728
\end{eqnarray*}
So
\begin{eqnarray*}
  Power_{new} = 95W\times\frac{3}{4}\times1.728\\
  =123.12W
\end{eqnarray*}

\section*{Problem 4 (6 points)}Identify all the data and name dependences (potential hazards) in the code below (specify if they are RAW, WAW, or WAR). Do not list indirect data dependences. State whether the dependence will cause a stall. Consider a five stage pipeline with IF ID EX MEM WB stages as in Appendix A.1. Branches are resolved in the ID stage. All stages take 1 cycle. Assume full forwarding.\\
1: ADD R1, R2, R3\\
2: LD R4, 0(R1)\\
3: ADD R1, R4, R5\\
4: SUB R4, R6, R7\\
5: BEQZ R4, done

\subsection*{Answer:}

\begin{tabular}{|l | l | l | l | l | l | l | l |}
  \hline
  1 & R2,R3 & R2+R3 &      & R1    &       &    &\\
  \hline
    & 2     & R1    & R1+0 & MEM   & R4    &    &\\
  \hline
    &       & 3     & R4,R5& R4+R5 &       & R1 &\\
  \hline
    &       &       & 4    & R6,R7 & R6+R7 &    & R4\\
  \hline
    &       &       &      & 5     & R4    &    &\\
  \hline
\end{tabular}

\begin{itemize}
\item Instruction 2 depends on 1 (RAW) because of R1. This will cause a stall.
\item 3 depends on 2 (RAW) because of R1. This will cause a stall.
\item 4 depends on 3 (WAR) because of R4. This will not cause a stall.
\item 5 depends on 4 (RAW) because of R4. This will cause a stall.
\end{itemize}

\section*{Problem 5 (10 points)}We will add support for register-memory ALU operations to the classic five stage RISC pipeline. To offset this increase in complexity, all memory addressing will be restricted to register indirect (i.e., all addresses are simply a variable held in a register: no offset or displacement may be added to the register value). For example, the register-memory instruction ADD R4,R5,(R1) means add the contents of register R5 to the contents of the memory location with address equal to the value in register R1 and put the sum in register R4. Register-register ALU operations are unchanged. Answer the following for the integer RISC pipeline.

\subsection*{a)} List a rearranged order of the five traditional stages of the RISC pipeline that will support register-memory operations implemented exclusively by register indirect addressing.

\subsection*{Answer:}

The new pipeline could be a 6 stage one, which is IF,ID,MEM1,EX,MEM2,WB. This design can support\\

ADD R4, R5, (R1)

as well as 

ADD (R1), R4, R5

\subsection*{b)} Describe what forwarding paths are needed for the rearranged pipeline by stating the source, destination, and when that path is used. Include forwarding paths which were also necessary in the original design.

\subsection*{Answer:}

\begin{itemize}
\item $MEM1/EX\to{}ID/MEM1$\\
LD R1, 0(R2)\\
STR R1, 0(R3)

\item $MEM1/EX\to{}IF/ID$\\
LD R1, 0(R2)\\
STR R1, 0(R3)\\
STR R1, 0(R4)

\item $EX/MEM2\to{}MEM1/EX$\\
ADD R1, R2, R3\\
ADD R4, R5, R1

\item $EX/MEM2\to{}ID/MEM1$\\
ADD R1, R2, R3\\
ADD R4, R5, R1\\
ADD R6, R7, R1

\item $EX/MEM2\to{}IF/ID$\\
ADD R1, R2, R3\\
ADD R4, R5, R1\\
ADD R6, R7, R1\\
ADD R8, R9, R1
\end{itemize}

\subsection*{c)}For the reordered stages of the RISC pipeline, what new data hazards are created by this addressing mode? Give an instruction sequence illustrating each new hazard.

\subsection*{Answer:}
Access of memory becomes a new kind of hazard.\\
\\
ADD (R1), R4, R5\\
SUB R6, R7, (R1)

\subsection*{d)}Show that that the RISC pipeline with register-memory ALU operations can take more or fewer instruction for a given program than the original RISC pipeline. Find an instruction sequence for the new architecture which is shorter than any equivalent sequences of instructions on the old architecture, and an instruction sequence for the old architecture which is shorter than any equivalent instruction sequence on the new architecture.

\subsection*{Answer:}

Program with shorter instruction sequence on new design:\\
\\
ADD R1, R4, (R5)\\
\\
LD R5, 0(R5)\\
ADD R1, R4, R5

Program with longer instruction sequence on new design:\\
\\
ADD R1, R1, 8\\
ADD R2, R2, 8\\
LD R1, 0(R1)\\
ADD R3, R1, (R2)\\
\\
LD R1, 8(R1)\\
LD R2, 8(R2)\\
ADD R3, R1, R2


\subsection*{e)}Assume all instructions take 1 clock cycle per stage. Many instructions are common to both architectures, but even compatible programs may have different performance. Give one compatible instruction sequence which will run with a higher CPI on the register-memory design than on the original pipeline, and one instruction sequence which will run with a lower CPI on the new design.

\subsection*{Answer:}

A sequence runs with a higher CPI on the new design:\\\
\\
LD R2, 0(R1)\\
LD R3, 0(R2)\\
LD R4, 0(R3)\\
\\
A sequence runs with a lower CPI on the new design:\\
ADD R1, R2, R3\\
ADD R4, R5, R6\\
ADD R7, R8, R9

\end{document}
