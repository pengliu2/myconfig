\documentclass[11pt,leqno]{article}
%define the title
\usepackage{amsmath}
\usepackage{fancyhdr}
\usepackage{listings}
\pagestyle{fancy}
\rhead{pengliu2}
\lhead{Peng Liu}
\author{Peng Liu(pengliu2, I2CS graduate)}
\title{Solutions for Homework 3}
\begin{document}
\lstset{language={[Motorola68k]Assembler}}
% generates the title
\maketitle
% insert the table of contents
%\tableofcontents
\section*{Problem 1}

\begin{enumerate}
\item[(a)]
  \begin{enumerate}
  \item[1.] SUB R1, R13, R14
  \item[2.] CMP.lt pT, pF = R1, R4
  \item[3.] (pF) ADDI R2, R2, \#1
  \item[4.] (pF) SW 0(R7), R2
  \item[5.] (pT) DIV.D F0, F0, F2
  \item[6.] (pT) ADD.D F0, F4, F2
  \item[7.] (pT) S.D 0(R8), F0
  \end{enumerate}

\item[(b)] 2 has an RAW on F1 from 1\\
  3 has an RAW on pF from 2\\
  4 has an RAW on pF from 2\\
  7 probably has a RAW on F0 from 6

\end{enumerate}

\section*{Problem 2}

\begin{enumerate}
\item[(a)] There are 22 cycles before the second iteration begins.
{\tiny
\begin{verbatim}
Loop:
  L.D F2, 0(R1)     IF ID EX ME WB
                       st
  ADD.D F2, F2, F2        IF ID EX EX EX ME WB
  L.D F4, 0(R2)              IF ID EX ME WB
                                st
  MUL.D F4, F4, F2                 IF ID EX EX EX EX EX EX EX ME WB
                                      st st st st st st
  ADD.D F6, F2, F4                                      IF ID EX EX EX ME WB
                                                           st
  S.D F6, 0(R3)                                               IF ID EX ME WB
  DADDUI R1, R1, #8                                              IF ID EX ME WB
  DADDUI R2, R2, #16                                                IF ID EX ME WB
  DADDUI R3, R3, #16                                                   IF ID EX ME WB
  DSUBUI R5, R5, #2                                                       IF ID EX ME WB
                                                                              st
  BNEZ R5, Loop                                                                  IF ID
                                                                                    st
\end{verbatim}
}

\item[(b)] There are 14 cycles after the loop is rescheduled.
{\tiny
\begin{verbatim}
Loop:
  L.D F2, 0(R1)     IF ID EX ME WB
  DADDUI R1, R1, #8    IF ID EX ME WB
  ADD.D F2, F2, F2        IF ID EX EX EX ME WB
  L.D F4, 0(R2)              IF ID EX ME WB
  DADDUI R2, R2, #16            IF ID EX ME WB
  MUL.D F4, F4, F2                 IF ID EX EX EX EX EX EX EX ME WB
  DSUBUI R5, R5, #2                   IF ID EX ME WB
  (DADDUI R7, R3, #0)                    IF ID EX ME WB
  DADDUI R3, R3, #16                        IF ID EX ME WB
  BNEZ R5, Loop                                IF ID EX ME WB
                                                  st st
  ADD.D F6, F2, F4                                      IF ID EX EX EX ME WB
  S.D F6, 0(R3)                                                  IF ID EX ME WB
\end{verbatim}
}

\item[(c)] Made one unrolling. 17 cycles before next iteration (the 3rd loop) begins.
{\tiny
\begin{verbatim}
Loop:
  L.D           F2, 0(R1)
  L.D           F12, 8(R1)
  ADD.D         F2, F2, F2
  ADD.D         F12, F12, F12
  L.D           F4, 0(R2)
  L.D           F14, 16(R2)
  MUL.D         F4, F4, F2
  MUL.D         F14, F14, F12
  DADDUI        R1, R1, #16
  DADDUI        R2, R2, #32
  DADDUI        R3, R3, #32
  DSUBUI        R5, R5, #4
  BNEZ          R5, Loop
  ADD.D         F6, F2, F4
  ADD.D         F16, F12, F14
  S.D           F6, -16(R3)
  S.D           F16, 0(R3)
  DADDUI        R1, R1, #32  
  DADDUI        R2, R2, #64 
  DADDUI        R3, R3, #64 
  DSUBUI        R5, R5, #8 
\end{verbatim}

}
  
\end{enumerate}

\section*{Problem 3}
\begin{enumerate}
\item[(a)] 10 cycles.
{\tiny
\begin{verbatim}
Loop:
        L.D     F2, 0(R1)
        L.D     F12, 8(R1)
        L.D     F22, 16(R1)
        L.D     F32, 24(R1)
        ADD.D   F2, F2, F2
        ADD.D   F12, F12, F12
        ADD.D   F22, F22, F22
        ADD.D   F32, F32, F32
        L.D     F4, 0(R2)
        L.D     F14, 16(R2)
        L.D     F24, 32(R2)
        L.D     F34, 48(R2)
        MUL.D   F4, F4, F2
        MUL.D   F14, F14, F12
        MUL.D   F24, F24, F22
        MUL.D   F34, F34, F32
        ADD.D   F6, F2, F4
        ADD.D   F16, F12, F14
        ADD.D   F26, F22, F24
        ADD.D   F36, F32, F34
        S.D     F6, 0(R3)
        S.D     F16, 16(R3)
        S.D     F26, 32(R3)
        S.D     F36, 48(R3)
        DADDUI  R1, R1, #32
        DADDUI  R2, R2, #64
        DADDUI  R3, R3, #64
        DSUBUI  R5, R5, #8
        BNEZ    R5, Loop
\end{verbatim}
}
{\tiny
\begin{tabular}{|l|l|l|l|l|}
  \hline
  Memory OP 1   & Memory OP 2   & Integer OP    & FP adder              & FP multiplier\\
  \hline
  \hline
  L.D F2,0(R1)  & L.D F12,8(R1) &               &                       & \\
  \hline
  L.D F4,0(R2)  & L.D F22,16(R1)&               & ADD.D F2,F2,F2        & \\
  \hline
  L.D F32,24(R1)& L.D F14,16(R2)&               & ADD.D F12,F12,F12     & MUL.D F4,F4,F2\\
  \hline
  L.D F24,32(R2) & L.D F34,48(R2)&              & ADD.D F22,F22,F22     & MUL.D F14,F14,F12\\
  \hline
                &               & DADDUI R1,R1,\#32& ADD.D F32,F32,F32  & MUL.D F24,F24,F22\\
  \hline
                &               & DADDUI R2,R2,\#64& ADD.D F6,F2,F4     & MUL.D F34,F32,F32\\
  \hline
  S.D F6,0(R3)  &               &               & ADD.D F16,F12,F14  & \\
  \hline
  S.D F16,16(R3)&               & DSUBUI R5,R5,\#8& ADD.D F26,F22,F24   & \\
  \hline
  S.D F26,32(R3)&               & DADDUI R3,R3,\#64& ADD.D F36,F32,F34     & \\
  \hline
  S.D F32,-16(R3)&              & BNEZ R5,Loop  &                       & \\

\end{tabular}
}
\item[(b)]
At least 3 loop unrollings are needed. Because there is only one FP adding unit, and each loop has two add operations. Other operations all can be parallelized. 

The first add operation has the RAW dependency on the first load. And the last store has the RAW dependency on the last add. So we need 2 more cycles. 

Thus if we unroll the loop for 3 times, total cycles would be $3\times 2 + 1 + 1 = 8$. If it's 2, $2\times 2 + 1 + 1 = 6$. But unrolling the loop 2 times can't create enough space to parallelize the instructions. 


\end{enumerate}
\section*{Problem 4}

\begin{enumerate}
\item[(a)] 
{\small
\begin{verbatim}
        S.D     F14, (R3)
        ADD.D   F14, F12, F11
        MUL.D   F11, F9, F8
        ADD.D   F8, F7, F7
        L.D     F7, 0(R1)
        L.D     F9, 0(R2)
        DADDUI  R1, R1, #8
        DADDUI  R2, R2, #16
        DADDUI  R3, R3, #16
        DSUBUI  R5, R5, #2
\end{verbatim}
}
\item[(b)]

Start-up code
{\small
\begin{verbatim}
        L.D     F7, 0(R1)
        L.D     F9, 0(R2)
        ADD.D   F8, F7, F7
        MUL.D   F11, F9, F8
        ADD.D   F14, F8, F11
        DADDUI  R1, R1, #8
        DADDUI  R2, R2, #16
        DSUBUI  R5, R5, #2

        L.D     F7, 0(R1)
        L.D     F9, 0(R2)
        ADD.D   F8, F7, F7
        MUL.D   F11, F9, F8
        DADDUI  R1, R1, #8
        DADDUI  R2, R2, #16
        DSUBUI  R5, R5, #2

        LDD.D   F7, 0(R1)
        LDD.D   F9, 0(R1)
        ADD.D   F8, F7, F7
        DADDUI  R1, R1, #8
        DADDUI  R2, R2, #16
        DSUBUI  R5, R5, #2
\end{verbatim}
}
Finish-up code
{\small
\begin{verbatim}
        S.D     F14, (R3)
        ADD.D   F14, F12, F11
        MUL.D   F11, F9, F8
        ADD.D   F8, F7, F7
        DADDUI  R3, R3, #16

        S.D     F14, (R3)
        ADD.D   F14, F12, F11
        MUL.D   F11, F9, F8
        DADDUI  R3, R3, #16

        S.D     F14, (R3)
        ADD.D   F14, F12, F11
        DADDUI  R3, R3, #16

        S.D     F14, (R3)
        DADDUI  R3, R3, #16
\end{verbatim}
}
\end{enumerate}

\section*{Problem 5}
\begin{enumerate}
\item[(a)]
To best hide its potential long latency, if possible, move the load instruction up as far as right after previous speculation check.
\begin{verbatim}
        instr.1
        instr.2
        ...
        sLD     F2, 0(R1)
        BEQZ    R1, null
        SPECCK  0(R1)
        ADD.D   F4, F0, F2
        J       merge
null:   ...
merge:  ...
\end{verbatim}

\item[(b)]
\begin{verbatim}
        instr.1
        instr.2
        ...
        sLD     F2, 0(R1)
        ADD.D   F4, F0, F2
        SPECCK  0(R1)
        J       merge
recover:
        restore old F4
null:   ...
merge:  ...
\end{verbatim}

\end{enumerate}
\section*{Problem 6}
\begin{enumerate}
\item[(a)] Bit 4-18 are used for index bits. 

\item[(b)] Bit 19-20 are used for cache tag.

\item[(c)] There are $2MB/16byte=131072 blocks$. And each block has a 2 bit tag and 1 state bit. So the total number of bits is $131072\times131=17170432$.

\end{enumerate}

\end{document}
