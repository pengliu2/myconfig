\documentclass[11pt,leqno]{article}
%define the title
\usepackage{amsmath}
\usepackage{fancyhdr}
\usepackage{listings}
\pagestyle{fancy}
\rhead{pengliu2}
\lhead{Peng Liu}
\author{Peng Liu(pengliu2, I2CS graduate)}
\title{Solutions for Homework 5}
\begin{document}
\lstset{language=C}
% generates the title
\maketitle
% insert the table of contents
%\tableofcontents
\section*{Problem 1}

\begin{enumerate}
\item[Part A] Solution:

The virtual address space \textbf{IS} larger than the physical address space.

The page offset in a virtual address is \textbf{21 bits long}.

The virtual page number is \textbf{43 bits long}.

\textbf{The most significant bits} is the page number in.

A physical page number is \textbf{27 bits long}.

The page offset in a physical address is \textbf{21 bits long}.

\textbf{16 pages or 32MB} is covered by a 16 entry TLB.

\textbf{0x7FFFC800089}

\item[Part B] Solution:
	\begin{enumerate}
	\item[A.]
	65 72 20 67
	\item[B.]
	There is one. It's virtual page 000
	\item[C.]
	There are two. They are virtual page 001 and 100
	\item[D.]
	There is none.
	\item[E.]
	73 61 79 20
	\item[F.]
	There will be an exception because virtual address is mapped to a physical page outside of installed physical memory. 
	\end{enumerate}

\end{enumerate}

\section*{Problem 2}
\begin{enumerate}
\item[Part A] Solution:
The 20 bits from bit 20 to bit 39 are used to obtain the translation.\\
Because the TLB is fully-associative, so these bits are to be compared with the tags of each TLB entry to address the TLB entry. And the physical page number is read from this TLB entry.

\item[Part B] Solution:

Bits 0-3 are used as block offset, directly from the virtual address.\\
Bits 4-11 are used as index, directly from the virtual address.\\
Bits 12-19 are used as tag.

\end{enumerate}

\section*{Problem 3}
\begin{enumerate}
\item[Part A] Solution:
The miss penalty is $8 + 50 + 25 \times 4 = 158$ cycles.

The CPI would be $1.5 + 158 \times 2\%= 4.66$.

\item[Part B] Solution:
The CPI would be $1.5 + 158 \time 80\% \time 2\% = 4.028$.

So the speedup would be $4.66/4.028 = 1.156901688$. 

\item[Part C] Solution:
The miss penalty would be $8 + 50 + 25 \time 2 = 108$ cycles.

CPI is $1.5 + 108 \times 2\% = 3.66$, which is faster than that in Part B.

\end{enumerate}

\section*{Problem 4}
	\begin{enumerate}
	\item[Part A] Solution:
	The answer is it depends on the size of a cache line. Given a cache line is x bytes long. Then the minimum associativity must be 
	\[
	2^{\log (128 \times 1K) - (12 - \log x) - \log x} 
	\]

	\item[Part B] Solution:
	
	Assume the L1 cache has to use 1 LSB to form a index. Then if the TLB has following entries in it, and the first words of both physical page 0 and 1 are in cache, then the cache will return incorrect data when the processor want to access the first word of either page.
	
	\begin{tabular}{|l|l|}
	Virtual Page Number & Physical Page Number \\
	\hline
	\hline
	1 & 0\\
	\hline
	0 & 1\\
	\hline
	\end{tabular}
	
	\item[Part C] Solution:
	The system can work in this way. The OS should ensure the 3 least significant bits of physical page number and virtual page number be identical when it'd mapping pages.
	
	\end{enumerate}

\end{document}
