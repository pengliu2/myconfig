\documentclass[11pt,leqno]{article}
%define the title
\usepackage{amsmath}
\usepackage{fancyhdr}
\pagestyle{fancy}
\rhead{pengliu2}
\lhead{Peng Liu}
\author{Peng Liu(pengliu2, I2CS graduate)}
\title{Solutions for Homework 2}
\begin{document}
% generates the title
\maketitle
% insert the table of contents
%\tableofcontents
\section*{Problem 1 (31 points)}
\subsection*{Part A [9 points]} 
\begin{tabular}{|l | l | l | l | l | l |}
  \hline
  NO.& Instruction      & IS    & EX    & WB    & Reason for Stalls\\
  \hline
  01 & L.D F0, 0(R1)    & 1     & 2     & 3     & \\
  \hline
  02 & ADD.D F2, F0, F4 & 2     & 4-8   & 9     & RAW on F0 (from 1)\\
  \hline
  03 & MUL.D F4, F2, F6 & 3     & 10-17  & 18    & RAW on F2 (from 2)\\
  \hline
  04 & ADD.D F6, F8, F10& 4     & 9-13  & 14    & Structural Hazard on EX (from 2)\\
  \hline
  05 & DADDI R1, R1, 8  & 5     & 6     & 7     & \\
  \hline
  06 & L.D F1, 0(R2)    & 6     & 7     & 8     & \\
  \hline
  07 & MUL.D F1, F1, F8 & 7     & 18-25 & 26    & Structure Hazard on EX (from 3)\\
  \hline
  08 & ADD.D F6, F3, F5 & 8     & 14-18 & 19    & Structure Hazard on EX (from 4)\\
  \hline
  09 & DADDI R2, R2, \#8& 9     & 10    & 11    & \\
  \hline
\end{tabular}

\subsection*{Part B [2 points]}
Pipelining some of the functional units does reduce the total execution time.\\
Pipelining the FP Multiplier unit does reduce the total cycles to 19.\\
If the FP Adder unit is also pipelined, the total cycles can be reduced to 18.

\subsection*{Part C [2 points]}
To add another FP Multiplier unit wouldn't make it better than pipelining it would do. Because 08 would be the last finished instruction no matter we pipeline the unit or add another one.

\subsection*{Part C [18 points]}

\begin{tabular}{|l | l | l | l | l | l | l |}
  \hline
  NO.& Instruction      & IS    & EX    & WB    & CM    & Reason for Stalls\\
  \hline
     &\textbf{Iteration 1}&       &       &       &       & \\
  \hline
  01 & LP: L.D F0, 0(R1)& 1     & 2     & 3     & 4     & \\
  \hline
  02 & ADD.D F0, F0, F6 & 1     & 4-8   & 9     & 10    & RAW on F0 (from 1)\\
  \hline
  03 & DIV.D F2, F2, F0 & 2     & 10-24 & 25    & 26    & RAW on F0 (from 02)\\
  \hline
  04 & L.D F0, 8(R1)    & 2     & 3     & 4     & 26    & \\
  \hline
  05 & DIV.D F4, F0, F8 & 3     & 25-39 & 40    & 41    & Wait for DIVDU\\
  \hline
  06 & S.D F4, 16(R1)   & 3     & 41    &       & 43    & RAW on F4 (from 05)\\
  \hline
  07 & DADDI R1, R1, \#-24&4     & 5     & 6     & 43    & \\
  \hline
  08 & BNEZ R1, LP      & 4     & 7     &       & 44    & RAW on F0 (from 07)\\
  \hline
     &\textbf{Iteration 2}&       &       &       &       & \\
  \hline
  09 & LP: L.D F0, 0(R1)& 5     & 7     & 8     & 44    & \\
  \hline
  10 & ADD.D F0, F0, F6 & 5     & 9-13  & 14    & 45    & RAW on F0 (from 09)\\
  \hline
  11 & DIV.D F2, F2, F0 & 6     & 40-54 & 55    & 56    & Wait for DIVDU\\
  \hline
  12 & L.D F0, 8(R1)    & 6     & 8     & 10    & 56    & Wait for DADDIU and CDB\\
  \hline
  13 & DIV.D F4, F0, F8 & 7     & 55-69 & 70    & 71    & Wait for DIVDU\\
  \hline
  14 & S.D F4, 16(R1)   & 11    & 12    & 71    & 72    & Wait for ROB and then DIVDU\\
  \hline
  15 & DADDI R1, R1, \#-24&27    & 28    & 29    & 72    & \\
  \hline
  16 & BNEZ R1, LP      & 27    & 30    &       & 73    & RAW on R1 (from 15)\\
  \hline
\end{tabular}


\section*{Problem 2 (14 points)}

\subsection*{Part A [4 points]}

4 correct predictions for branch 1, 2 for branch 2.

\begin{tabular}{|l | l | l |}
  \hline
  Step  & Branch 1 Prediction   & Actual Branch 1 Action\\
  \hline
  1     & N                     & T\\
  \hline
  2     & T                     & T\\
  \hline
  3     & T                     & N\\
  \hline
  4     & N                     & T\\
  \hline
  5     & T                     & T\\
  \hline
  6     & T                     & N\\
  \hline
  7     & N                     & T\\
  \hline
  8     & T                     & T\\
  \hline
  9     & T                     & N\\
  \hline
  10     & N                     & T\\
  \hline
  11     & T                     & T\\
  \hline
  12     & T                     & N\\
  \hline
\end{tabular}

\begin{tabular}{|l | l | l |}
  \hline
  Step  & Branch 2 Prediction   & Actual Branch 1 Action\\
  \hline
  1     & N                     & T\\
  \hline
  2     & T                     & T\\
  \hline
  3     & T                     & T\\
  \hline
  4     & T                     & N\\
  \hline
\end{tabular}

\subsection*{Part B [4 points]}

6 correct prediction for branch 1. 1 for branch 2.

\begin{tabular}{|l | l | l | l |}
  \hline
  Step  & Counter Value & Branch 1 Prediction   & Actual Branch 1 Action\\
  \hline
  1     & 00            & N                     & T\\
  \hline
  2     & 01            & N                     & T\\
  \hline
  3     & 11            & T                     & N\\
  \hline
  4     & 10            & T                     & T\\
  \hline
  5     & 11            & T                     & T\\
  \hline
  6     & 11            & T                     & N\\
  \hline
  7     & 10            & T                     & T\\
  \hline
  8     & 11            & T                     & T\\
  \hline
  9     & 11            & T                     & N\\
  \hline
  10    & 10            & T                     & T\\
  \hline
  11    & 11            & T                     & T\\
  \hline
  12    & 11            & T                     & N\\
  \hline
\end{tabular}

\begin{tabular}{|l | l | l | l |}
  \hline
  Step  & Counter Value & Branch 2 Prediction   & Actual Branch 1 Action\\
  \hline
  1     & 00            & N                     & T\\
  \hline
  2     & 01            & N                     & T\\
  \hline
  3     & 11            & T                     & T\\
  \hline
  4     & 11            & T                     & N\\
  \hline
\end{tabular}

\subsection*{Part C [6 points]}

4 correct prediction for Branch 1, 2 for branch 2.

\begin{tabular}{|l | l | l | l |}
  \hline
  Step  & Branch 1 Prediction   & Actual Branch 1 Action& New State\\
  \hline
  1     & N                     & T                     & N/N/N/T\\
  \hline
  2     & N                     & T                     & N/T/N/T\\
  \hline
  3     & N                     & N                     & N/T/N/T\\
  \hline
  4     & T                     & T                     & N/T/N/T\\
  \hline
  5     & N                     & T                     & T/T/N/T\\
  \hline
  6     & T                     & N                     & N/T/N/T\\
  \hline
  7     & T                     & T                     & N/T/N/T\\
  \hline
  8     & N                     & T                     & T/T/N/T\\
  \hline
  9     & T                     & N                     & N/T/N/T\\
  \hline
  10    & T                     & T                     & N/T/N/T\\
  \hline
  11    & N                     & T                     & T/T/N/T\\
  \hline
  12    & T                     & N                     & N/T/N/T\\
  \hline
\end{tabular}

\begin{tabular}{|l | l | l | l |}
  \hline
  Step  & Branch 2 Prediction   & Actual Branch 1 Action& New State\\
  \hline
  1     & N                     & T                     & N/N/T/N\\
  \hline
  2     & T                     & T                     & N/N/T/N\\
  \hline
  3     & T                     & T                     & N/N/T/N\\
  \hline
  4     & T                     & N                     & N/N/N/N\\
  \hline
\end{tabular}


\section*{Problem 3 [12 points]}
\subsection*{Part A}
The history buffer would have an extra field to store the old values, while the reorder buffer doesn't need this.

\subsection*{Part B}
In the Issue stage, the register value before the instruction issues need to be appended in the history buffer. 

\subsection*{Part C}
There is no COB, and the Execute stage uses bypass path for some of the operands.

\subsection*{Part D}
Complete stage is not necessary. 

\subsection*{Part E}
Exception flag is checked in the Commit stage. If this instruction caused an exception, the entry in history buffer need to be marked.

If there is no exception caused by this instruction, the instruction is marked as finished in history buffer.

Memory can hardly be reverted, so stores have to commit to memory in order. Before committing to memory, a store has to make sure all previous ones has committed. 

Like stores, branches are not easy to be reverted, so they have to ensure previous branches have committed. 

If the instruction is on the top of the history buffer, it is removed from the buffer. Or it's marked as finished.

\subsection*{Part F}
When an exception happens, the instruction which causes the instruction should continue. When it finishes, the exception is marked at the corresponding entry in history buffer.  

When an instruction with exception flag moves to the top of history buffer, all following instructions are removed from the buffer, and register value rolls back.

We don't care about the preciseness of external interrupts.


\end{document}
