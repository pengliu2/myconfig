% LaTeX resume using res.cls
\documentclass{res}
%\usepackage{helvetica} % uses helvetica postscript font (download helvetica.sty)
\usepackage{newcent}   % uses new century schoolbook postscript font
\usepackage{ulem}
\setlength{\topmargin}{-0.6in}  % Start text higher on the page
\setlength{\textheight}{9.8in}  % increase textheight to fit more on a page
\setlength{\headsep}{0.2in}     % space between header and text
\setlength{\headheight}{12pt}   % make room for header
\usepackage{fancyhdr}  % use fancyhdr package to get 2-line header
\renewcommand{\headrulewidth}{0pt} % suppress line drawn by default by fancyhdr
\newsectionwidth{0pt} % So the text is not indented under section headings

\lhead{\hspace*{-\sectionwidth}Peng Liu} % force lhead all the way left

\rhead{Page \thepage}  % put page number at right
\cfoot{}  % the footer is empty
\pagestyle{fancy} % set pagestyle for the document

\begin{document}
\thispagestyle{empty} % this page does not have a header

\name{LIU, PENG}
% \address used twice to have two lines of address
\address{111 South Busey Avenue Apt 5, Urbana, IL 61801\\
  (217)898-3636\\
  pengliu2@illinois.edu}

\begin{resume}
  \vspace{0.1in}

  \section{OBJECTIVE}
  \vspace{0.1in}
  Looking for a position as a software engineer where I can utilize my skills in Linux programming and experiences in embedded device development to be a valuable team player in your organization.

  \section{EXPERIENCE}
  \uline{{\sl Sr. Software Engineer, Android Kernel/BSP team of Motorola}} \hfill \uline{Feb 2006 - Present}

  With C, Python, Perl and Java, implementing smart phone features and development tools. Leading the team in some projects.

  \begin{itemize}
  \item {{\sl Tools and methods for energy analysis on ARM Linux }} \hfill \texttt{C on ARM Linux}\\
    {\sl Role: Owner}                                               \hfill \texttt{Python and C++ on Windows}

    The goal is to find out the software activities which are consuming battery energy the most, and, to provide logs and profiles to support power management optimization. The approach includes automatically current drain measuring, power event tracing, OS event tracing, and integrally log analyzing. 

    This is playing a key role in the development for latest flagship products. Following jobs have been done to build up a set of systematic methods to keep track of energy usage, and to point out the violator and optimize battery performance.

    \begin{itemize}
    \item Complemented Linux kernel 2.6.35 with support for tracing OMAP4 power events. The upstream Linux kernel of this version doesn't support tracing these for ARM, so I back ported it. With this ported, CPU frequency adjustments, CPU hotplug events, CPU idle state changes and all other power related events can be logged. Logging can be turn on and off without phone reboot, and its overhead is low.
    \item Implemented a current drain measurement tool. It formalized and automated the measurement. The objectives now include idle mode current drain, audio playback current drain, video playback current drain, etc, and can be defined by the developers. 
    \item Created a set of tools to analyze traces. The amount of logs can be huge, so I developed helper applications to accelerate the analysis, for instance, the tool to analyze Linux kernel logs and user space logs together, the tool to visualize CPU idle mode switching, and the tool to create stats for wake-ups events. 
    \item Discovered several reliable methods to analyze the logs. These methods use the logs, the tools and the knowledge on Linux to guide the system optimization from the power management point of view.
    \end{itemize}

  \item {\sl Debug infrastructure}                                   \hfill \texttt{C/Assembly Language on ARM Linux}\\
    {\sl Role: Leader}

    This is the Motorola's unique solution to capture logs for Android phones, covering from Linux kernel to applications.

    \begin{itemize}
    \item Designed and implemented a method to capture the snapshot of system before watchdog reset issues. This helped BSP team fix complicated issues leading to watchdog reset. The implementation included deeply hacking into ARM Monitor mode code, bootloader and Linux kernel.
    \item Application Processor logger. It serializes Android logs and Linux kernel messages onto non-volatile memory. It has become one of the most important tool to undertake off-line analysis.
    \item Ported driver and parsing tool for ARM CoreSight Trace Macrocells to accelerate debugging low level software issues.
    \item Helper tools, such as log merge script and visualizing script.
    \end{itemize}

  \item {\sl PCK, a package management system for embedded system development.} \hfill \texttt{Jave on Eclipse}\\
    {\sl Role: Leader}

    It has been used in iDEN product line by over 400 developers. Functionality includesed development environment setting-up, system assembling, image generating and signature generating.
  \end{itemize}

  \uline{{\sl Software Engineer, Pollex Mobile Software}} \hfill                  \uline{Aug 2003 - Jan 2006}

  Developed major parts of Hopen Operating System 3.0, a proprietary RTOS. Ported Hopen OS to different CPUs, including Intel XScale PXA270 and IBM PowerPC 405. 

  \begin{itemize}
  \item {\sl Hopen real-time OS }                                      \hfill \texttt{C/Assembly Language on ARM Linux}\\
  {\sl Role: Key Developer}

  Hopen OS has been successfully deployed on more than 10 products of leading mobile device vendors. It was also certificated as one of best innovative products by Beijing Municipal Science and Technology Committee. My job was developing major parts, porting it to deferent hardware.
  \end{itemize}

  \uline{{\sl Research Assistant, State Key Lab of Fire Science of China}} \hfill                 \uline{Sep 2000 - Jul 2003}

  Participated in the research project, the application of Ethernet in real-time
  control system. As a programmer, participated in the project to construct a ceramics
  manufacture control system, designed and implemented a CAN field bus interface
  circuit board and its WDM driver.


  \section{SOFTWARE SKILLS}
  \vspace{0.1in}

  {\sl Languages}: \texttt{C/C++, ARM assembly, Python, Java, Perl, Virtual Basic for Applications}\\
  {\sl Platforms}: \texttt{Linux kernel, Android, Qt, WDM SDK, Eclipse, NI-DAQmx}\\
  {\sl Software}: \texttt{Git, Emacs, Vim, Clearcase, Foswiki, Hudson, Eclipse, MS Office(Excel, Word)}


  \section{EDUCATION} {\sl}
  \vspace{0.1in}
  % \sl will be bold italic in New Century Schoolbook (or
  % any postscript font) and just slanted in
  % Computer Modern (default) font
  Master of Computer Science(off-campus) \hfill 2010-Present\\
  University of Illinois, Urbana, IL

  M.E. in Safety Engineering \hfill 2000-2003\\
  University of Science and Technology of China, Hefei, China

  B.E. in Automation \hfill 1995-2000\\
  University of Science and Technology of China, Hefei, China
  
\end{resume}
\end{document}

