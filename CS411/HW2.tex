\documentclass[letter,11pt]{article}
\usepackage{fancyhdr}
\usepackage{graphicx}
\usepackage{listings}
\usepackage{ulem}
\usepackage{amsmath}
\pagestyle{fancy}
\rhead{pengliu2}
\lhead{Peng Liu}
\author{Peng Liu, I2CS student}
\title{Homework 2\\CS411 2011 Spring}

\begin{document}
\lstset{language=SQL}
\maketitle

\section*{Problem 1: ER Diagram Design}Solution

1. $\pi_{char\_id}(\sigma_{username='Beowulf'}(Account\bowtie Characters))$

2. $\pi_{account\_id}-\pi_{account\_id}(\sigma_{realmName\neq'Thor'}(Server\bowtie Characters))$

3. $\pi_{character\_name}(\sigma_{character\_class='Warrior'\ and\ realName\neq'Asgard'}(Server\bowtie Chracters))$

4. $\pi_{username}(Account)-\pi_{username}(\sigma_{level<100}(Account\bowtie Characters)$

\section*{Problem 2: Thinking about Relational Algebra}Solution

\begin{enumerate}
\item[1]
  \begin{enumerate}
  \item[1] Yes. They have the same output.
  \item[2] Yes. They have the same output.
  \item[3] No. For example, R(a,b)={(1,2),(2,3)} and S(a,b)={(2,3),(3,4)}\\
        S-R = {(3,4)}\\
        R-S = {(1,2)}
  \end{enumerate}

\item[2]
  \begin{enumerate}
  \item[1] The minimum number of tuples of $R\cup S$ is max(n,m).\\
    The maximum number is n+m.
  \item[2] $R\bowtie_C S$ :\\
    Minumum: 0\\
    Maximum: nxm
  \item[3] R - S :\\
    Minumum: 0 if $n<m$ or $|n-m|$\\
    Maximum: n
  \end{enumerate}
\end{enumerate}

\section*{Problem 3: More Relational Algebra}Solution

\begin{enumerate}
\item[1]
  \begin{enumerate}
  \item[1.1] Violates
  \item[1.2] Does not violate
  \item[1.3] Violates
  \end{enumerate}
\item[2]
  \begin{enumerate}
  \item[2.1] T(a,b,b',c)={(w,x,w,x),(w,x,w,y),(w,x,x,y),(y,z,w,x),(y,z,w,y),(y,z,x,y)}
  \item[2.2] T(a,b,c)={(w,x,y)}
  \item[2.3] T(a,b,b',c)={(w,x,w,x),(w,x,w,y),(w,x,x,y)}
  \end{enumerate}
\end{enumerate}

\section*{Problem 4: Functional Dependencies}Solution
\begin{enumerate}
\item Closures are:
  \begin{enumerate}
  \item (B):(B,D)
  \item(A,B):(A,B,D)
  \item (A,C):(A,B,C,D)
  \end{enumerate}
\item (minimal)keys are (A,C) and (B,C). Because:
  \begin{itemize}
  \item nothing can determine C, C must be in any key. 
  \item Considering one element subsets, (C) is not a key because it can't determine A and B.
  \item In two element subsets, (A,C) and (B,C) are (minimal)keys, because A, B or C is not a key, thanks to above analysis.
  \end{itemize}
\item (super)keys that are not (minimal)keys are:

  (A,B,C)

  (A,C,D)

  (B,C,D)

  (A,B,C,D)

\item

  $AC\to B$

\end{enumerate}

\end{document}


